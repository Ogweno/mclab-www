% This file was converted to LaTeX by Writer2LaTeX ver. 1.0.2
% see http://writer2latex.sourceforge.net for more info
\documentclass[letterpaper]{article}
\usepackage[ascii]{inputenc}
\usepackage[T1]{fontenc}
\usepackage[english]{babel}
\usepackage{amsmath}
\usepackage{amssymb,amsfonts,textcomp}
\usepackage{color}
\usepackage{array}
\usepackage{hhline}
\usepackage{hyperref}
\hypersetup{pdftex, colorlinks=true, linkcolor=blue, citecolor=blue, filecolor=blue, urlcolor=blue, pdftitle=, pdfauthor=Vineet Kumar, pdfsubject=, pdfkeywords=}
% List styles
\newcommand\liststyleLi{%
\renewcommand\theenumi{\arabic{enumi}}
\renewcommand\theenumii{\arabic{enumii}}
\renewcommand\theenumiii{\arabic{enumiii}}
\renewcommand\theenumiv{\arabic{enumiv}}
\renewcommand\labelenumi{\theenumi.}
\renewcommand\labelenumii{\theenumii.}
\renewcommand\labelenumiii{\theenumiii.}
\renewcommand\labelenumiv{\theenumiv.}
}
\newcommand\liststyleLii{%
\renewcommand\theenumi{\arabic{enumi}}
\renewcommand\theenumii{\arabic{enumii}}
\renewcommand\theenumiii{\arabic{enumiii}}
\renewcommand\theenumiv{\arabic{enumiv}}
\renewcommand\labelenumi{\theenumi.}
\renewcommand\labelenumii{\theenumii.}
\renewcommand\labelenumiii{\theenumiii.}
\renewcommand\labelenumiv{\theenumiv.}
}
\newcommand\liststyleLiii{%
\renewcommand\theenumi{\arabic{enumi}}
\renewcommand\theenumii{\arabic{enumii}}
\renewcommand\theenumiii{\arabic{enumiii}}
\renewcommand\theenumiv{\arabic{enumiv}}
\renewcommand\labelenumi{\theenumi.}
\renewcommand\labelenumii{\theenumii.}
\renewcommand\labelenumiii{\theenumiii.}
\renewcommand\labelenumiv{\theenumiv.}
}
% Page layout (geometry)
\setlength\voffset{-1in}
\setlength\hoffset{-1in}
\setlength\topmargin{2cm}
\setlength\oddsidemargin{2cm}
\setlength\textheight{23.94cm}
\setlength\textwidth{17.59cm}
\setlength\footskip{0.0cm}
\setlength\headheight{0cm}
\setlength\headsep{0cm}
% Footnote rule
\setlength{\skip\footins}{0.119cm}
\renewcommand\footnoterule{\vspace*{-0.018cm}\setlength\leftskip{0pt}\setlength\rightskip{0pt plus 1fil}\noindent\textcolor{black}{\rule{0.25\columnwidth}{0.018cm}}\vspace*{0.101cm}}
% Pages styles
\makeatletter
\newcommand\ps@Standard{
  \renewcommand\@oddhead{}
  \renewcommand\@evenhead{}
  \renewcommand\@oddfoot{}
  \renewcommand\@evenfoot{}
  \renewcommand\thepage{\arabic{page}}
}
\makeatother
\pagestyle{Standard}
\title{}
\author{Vineet Kumar}
\date{2012-06-06}
\begin{document}

\bigskip

Tutorial about how to build the environment and run samples


\bigskip

Get your own copy of the software:


\bigskip

\liststyleLi
\begin{enumerate}
\item Jar package : Click here{\textless}!-{}-link to the
jar-{}-{\textgreater} to download the compiled Jar package

OR
\item Source Code : McLab source code is freely provided under the
Apache Version 2.0 license. Go to
\url{http://www.sable.mcgill.ca/mclab/download_mclab.html} download the
source code as a .tar.gz source archive for UNIX-like systems or a .zip
source archive for Windows.
\end{enumerate}

\bigskip

{\textless}!-{}-

P.S. You can also generate the jar package after you download the source
code with ant: go to the directory of \~{}{\textbackslash}Matlab
Source{\textbackslash}languages{\textbackslash}Natlab, you will find
there is a build.xml file. You can compile the source code to jar
package in Eclipse by adding the build file into ant window and double
click the option of jar, or you can type the command as
{\textquotedblleft}ant jar{\textquotedblright} in your command line
when you are under the directory of \~{}{\textbackslash}Matlab
Source{\textbackslash}languages{\textbackslash}Natlab.

{}-{}-{\textgreater}


\bigskip

Build from the source code 


\bigskip

Building on UNIX-like systems


\bigskip

\liststyleLii
\begin{enumerate}
\item Make sure you have JDK6(or higher) and ANT installed on your
system
\item Unpack the .tar.gz source archive
\item run cd ./Matlab{\textbackslash} Source/languages/Natlab
\item run ant jar
\item If everything goes well, you should get a Natlab.jar package


\bigskip

Building on Windows
\end{enumerate}

\bigskip

\liststyleLiii
\begin{enumerate}
\item Make sure you have JDK6(or higher) and ANT installed on your
system
\item Unpack the .zip source archive
\item Open a Command window (Select Start-{\textgreater} Run, type
{\textquotedblleft}cmd{\textquotedblright} and press Enter)
\item Go to the directory where you have downloaded the .zip archive
\item run cd {\textquotedblleft}Matlab
Source{\textbackslash}languages{\textbackslash}Natlab{\textquotedblright}
\item run ant build
\end{enumerate}

\bigskip

Import the mclab source code into the workspace, the eclipse will
compile a new folder named bin-eclipse automatically. But this is not
enough, we have to compile the source code by ourselves. Open ant
window in eclipse and add new build.xml by click {\textquotedblleft}Add
Buildfiles{\textquotedblright}. And then, double click
{\textquotedblleft}build [default]{\textquotedblright}, this operation
will build the environment for you. If the building process is
successful, you will find a folder named
{\textquotedblleft}bin-ant{\textquotedblright} under
\~{}{\textbackslash}Matlab
Source{\textbackslash}languages{\textbackslash}Natlab directory. And
wait a few seconds, you will find there will be no error in the Natlab
source code in the eclipse workspace and now, you can run it with some
certain command. And the main file (entry point) of Natlab project is
under \~{}{\textbackslash}Matlab
Source{\textbackslash}languages{\textbackslash}Natlab{\textbackslash}src{\textbackslash}natlab
directory.


\bigskip

2.Command line in Windows:

The prerequisite of building the environment is you should have both
Java JDK and ANT in your computer. And here, we just give the direction
of how to install ant and set environment variables:

1.First, you must have Java JDK installed in your computer;

2.Get the ANT for windows from the Internet,
http://ant.apache.org/bindownload.cgi and pick the .zip archive, for
example, the latest one is apache-ant-1.8.3-bin.zip;

3.After download ANT, extract zip package in (for example)
C:{\textbackslash}Ant;

4.Set ANT\_HOME

[F0A7?]Right click My Computer icon

[F0A7?]Choose properties

[F0A7?]Choose Advanced Tab

[F0A7?]Choose Environment Variables Button

[F0A7?]In the System Variables, click New Button

[F0A7?]Give the Variable Name: ANT\_HOME, and give the Value:
C:{\textbackslash}Ant

[F0A7?]Click OK

Then, we{\textquoteright}ll add new ANT\_HOME path,

Find PATH (or Path) in the Variable Column in System variables frame

[F0A7?]After found, click Edit button

[F0A7?]Then, add
{\textquotedblleft}\%ANT\_HOME\%{\textbackslash}bin;{\textquotedblright}
in the bottom of Variable value

[F0A7?]Click OK to finish

5.Check whether ANT works correctly or not

In the command prompt, type {\textquotedblleft}ant
--version{\textquotedblright}, then click Enter. If the result text is
something like this, {\textquotedblleft}Apache Ant(TM) version 1.8.3
compiled on February 26 2012{\textquotedblright}, then your ANT works
correctly on your Windows.

The reference of this section is from
http://omrudi.wordpress.com/2008/11/08/how-to-install-ant-in-windows-xp/.

After installing the ANT, we go to the directory of
\~{}{\textbackslash}Matlab
Source{\textbackslash}languages{\textbackslash}Natlab, type
{\textquotedblleft}ant build{\textquotedblright} in your Windows
command prompt. Then the environment of the project is built.


\bigskip

Run the software

There are two options to run the software:

1.Compiling the source code into jar package, and run it in command
line. For example, after compiling the source code, there will be a jar
package named Natlab.jar under the directory of
\~{}{\textbackslash}Matlab
Source{\textbackslash}languages{\textbackslash}Natlab, open the command
prompt, go to the same directory, type {\textquotedblleft}java --jar
Natlab.jar -help{\textquotedblright}, you will get the options of
Natlab. [give some examples here, like --x and --t, or make a new
section below?]

2.You can also run the software in Eclipse. Open {\textquotedblleft}Run
Configurations{\textquotedblright}, click
{\textquotedblleft}Search{\textquotedblright} in
{\textquotedblleft}Main class{\textquotedblright} section, pick
{\textquotedblleft}Main - natlab{\textquotedblright}. Then go back to
{\textquotedblleft}Run Configurations{\textquotedblright} panel, click
{\textquotedblleft}Arguments{\textquotedblright} and type command in
{\textquotedblleft}Program arguments{\textquotedblright} section. For
example, type {\textquotedblleft}-x
\~{}/test/test.m{\textquotedblright} into the section. Then, click
{\textquotedblleft}run{\textquotedblright} button at the right bottom
corner.

According to my own experience, we suggest you to choose the first
option to run the software. And if you want to know exactly how the
program works, then you can go to the source code.


\bigskip

FAQ:

1.When you compile the code, if the Eclipse throws some error like this:


{\textquotedblleft}com.sun.tools.javac.Main is not on the classpath.
Perhaps JAVA\_HOME does not point to the JDK. It is currently set to
{\textquotedbl}C:{\textbackslash}Program
Files{\textbackslash}Java{\textbackslash}jre6{\textquotedbl}{\textquotedblright}

First, check whether you install the JDK on your machine, because
Windows people always use Eclipse with JRE to write programs. If you
have already installed the JDK, copy the tools.lib in the directory of
jdk{\textbackslash}lib to the directory of jre{\textbackslash}lib.

2.In Windows, remember the file path should not have white space, if
there is white space, use double quotation to quote the path and file,
like this: java --jar Natlab.jar -- x
{\textquotedblleft}D:{\textbackslash}java{\textbackslash}workspace{\textbackslash}Matlab
Source{\textbackslash}test.m{\textquotedblright}
\end{document}
