\begin{otherlanguage}{french} 
\matlab est un langage de programmation dynamique, orienté-tableaux
communément utilisé par les étudiants, les scientifiques et les
ingénieurs qui apprécient son style de développement interactif, la
richesse de ses opérateurs sur les tableaux, sa librairie
impressionnante de fonctions de base et le fait qu'on aie pas à
déclarer statiquement le type des variables.  Bien que ces usagers
apprécient \matlab, leurs programmes nécessitent souvent des ressources
de calcul importantes qui sont offertes par les nouveaux systèmes de
haute performance.  Cette thèse fait le rapport de \mixten, un
compilateur source-à-source qui fait la traduction automatique de
programmes \matlab à \xten, un langage construit pour "la performance et
la productivité à grande échelle."  Ainsi, \mixten aide les programmeurs
scientifiques à faire un meilleur usage des ressources des systèmes de
calcul de haute performance.

Il y a un écart sémantique important entre le typage dynamique et le
focus sur les tableaux de \matlab et l'approche orientée-objet, le
typage statique et les abstractions de haut niveau sur les tableaux de
\xten.  Cette thèse discute des défis principaux qui doivent être
surmontés afin de produire du code \xten séquentiel compétitif avec les
meilleurs compilateurs statiques pour \matlab qui traduisent vers des
langages impératifs plus conventionnels, tels que C et Fortran.  Fort
de cette fondation efficace, cette thèse décrit ensuite la traduction
de l'instruction \texttt{parfor} de \matlab afin d'utiliser les opérations
sophistiquées de traitement concurrent de \xten.

Le compilateur \mixten a été implémenté à l'aide de la suite d'outils de
McLab, un projet libre de droits, disponible à la communauté de
recherche ainsi qu'aux utilisateurs de \matlab.  Nous avons utilisé
notre implémentation afin d'effectuer des mesures empiriques de
performance sur un jeu de 17 programmes \matlab.  Nous démontrons que
le code généré par \mixten est considérablement plus rapide que le
système \matlab de Mathworks et que nos résultats sont compétitifs avec
les meilleurs compilateurs statiques qui produisent du code C et
Fortran.  Nous montrons également l'importance d'une représentation
appropriée des tableaux en \xten et la nécessité d'une analyse
\emph{IntegerOkay} qui permet de déterminer quelles variables de type réel
(double) peuvent être correctement représentées par des entiers
(int). Finalement, nous montrons que notre traitement de l'instruction
\texttt{parfor} en \xten nous permet d'atteindre des vitesses d'exécution
considérablement meilleures que dans \matlab.

\end{otherlanguage}

